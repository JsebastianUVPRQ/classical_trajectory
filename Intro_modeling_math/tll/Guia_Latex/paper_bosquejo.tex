\documentclass[twocolumn]{article}  
\usepackage[utf8]{inputenc}  
\usepackage[spanish]{babel}  
\usepackage{graphicx}  
\usepackage{amsmath} % matematica real 
\usepackage{hyperref}  

\title{Modelo SIR}  
\author{Autor 1 \\ \textit{Universidad del Valle-Cali}\\ Curso: IMM}  
\date{12 de noviembre de 2024}  

\begin{document}  

\maketitle  

\begin{abstract}  
  Este es un resumen del artículo, donde se describen brevemente los objetivos, metodologías y resultados principales. Debe ser conciso y claro.  
\end{abstract}  

\section{Introducción}  
La introducción debe contener una revisión del tema a tratar, la relevancia del mismo y el objetivo del artículo. Aquí deberías incluir citas bibliográficas si es necesario y establecer el contexto del trabajo.
\cite{smith2004sir}

\section{Metodología}  
Describir la metodología utilizada en la investigación. Esto puede incluir busqueda de codigos, documentos revisados y técnicas de análisis. Es importante ser lo más claro y específico posible.  

\section{Resultados}  
Presentar los resultados obtenidos. Puedes incluir tablas y gráficos para facilitar la comprensión. Asegúrate de interpretar los datos de manera que se vinculen con los objetivos planteados en la introducción.  

\includegraphics[scale=0.4]{Ejemplo.pdf}
\section{Discusión}  
En esta sección, discute los resultados obtenidos, comparándolos con otros estudios previos. Aquí es donde se puede considerar la importancia de los hallazgos y las posibles implicaciones.  

\section{Conclusiones}  
Resumir los hallazgos más importantes del estudio y su relevancia. Puedes incluir recomendaciones para investigaciones futuras o aplicaciones prácticas de los resultados.  

\section*{Agradecimientos}  
Agradecer a las personas o instituciones que contribuyeron al desarrollo de la investigación, si corresponde.
 \bibliographystyle{elsarticle-num}
 \bibliography{Biblio}
%\bibliographystyle{plain}
%\bibliography{referencias} % Si tienes un archivo .bib para tus referencias  

\end{document}